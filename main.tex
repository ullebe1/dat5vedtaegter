\documentclass{article}
\usepackage[utf8]{inputenc}
\usepackage{titlesec}
\usepackage{hyperref}
\usepackage[
type={CC},
modifier={by-nc-sa},
version={4.0},
]{doclicense}

\title{Samlede Vedtægter}
\author{De forenede studerende på Dat5}
\date{February 2016}

\newcommand{\KTD}{Kamp til Døden\textsuperscript{TM}}
\newcommand{\JJB}{Jul Jul Binks}

\let\OldS\S
\renewcommand{\S}{\OldS{}}

\titleformat{\section}
{\normalfont\Large\bfseries}{\S\thesection}{1em}{}
\titleformat{\subsection}
{\normalfont\large\bfseries}{Stk. \thesubsection}{1em}{}
\titleformat{\subsubsection}
{\normalfont\large\bfseries}{Stk. \thesubsubsection}{1em}{}
\def\thesubsection{\arabic{subsection}}
\def\thesubsubsection{\arabic{subsection}.\arabic{subsubsection}}

\begin{document}

	\maketitle

	\section{Holdets medlemmer}
	Som medlem af holdet betragtes dem, som har eller har haft teoretiske øvelser på holdet DA5 på datalogistudiet på Aarhus Universitet på årgangen, som havde sin begyndelse i august 2014.

	\section{Vedtægter}
	\subsection{Gældende version af vedtægterne}
	Den gældende version af vedtægterne kan altid findes som kildekode på\\ https://github.com/ullebe1/dat5vedtaegter som pdf på\\ https://www.sharelatex.com/github/repos/ullebe1/dat5vedtaegter/.
	\subsection{Vedtægtsændringer}
	Vedtægtsændringer kan foretages ved afstemning ved holdets time og forslag kan indsendes som pull-requests på github.

	\section{Komunikation}
	Holdets primære fælles kommunikationsplatform er facebookgruppen ”AU DAT 5 2014”, og som findes på https://www.facebook.com/groups/AUDAT514/. Ud over medlemmer af DA5 har Astrid 'FUIS' Christiansen og Camilla 'SOVS' Grevy adgang til gruppen.

	\section{Afstemninger}
	Alle afstemninger afgøres enten ved håndsoprækning og almindeligt flertal ved holdets time eller ved et spin med \JJB. Afstemninger kan have så mange mulige udfald, som initiativtageren til afstemningen synes. Dog mindst 2. Da afstemninger er d'ary, kan alle stemme på alle udfald. Er der stemmelighed løses dette ved \KTD.
	\subsection{Stemmeret}
	Alle som betragtes som fra samme årgang som DA5 har automatisk stemmeret. Personer som er stedfortrædere for personer på DA5 har samme stemmeret som den person de er stedfortræder for. Det er op til ordstyreren at tælle stemmer og holde styr på hvem som har og ikke har stemmeret.
	\subsubsection{Vetoret}
	Uanset hvad et flertal stemmer igennem til holdets time, så har et flertal af tilstedeværende medlemmer på DA5 mulighed for at nedlægge veto.

	\section{Ansvarsposter}
	\subsection{Kasserer}
	Holdet udpeger en kasserer. Kassererens primære opgave er at være ansvarlig forholdets bødekasse. Kassereren udpeges for ét år ad gangen, og kan godt genvælges.
	\subsection{Tuschansvarlig}
	Da det ofte kan være svært at finde en whiteboardtusch, indkøber holdet en tusch og udpeger en tuschansvarlig. Det er den tuschansvarliges fornemmeste opgave at opbevare tuschen sikkert, og sørge for at den altid er til stede ved holdets time. Skulle
	tuschen mod forventning bortkomme, er det den tuschansvarliges opgave at anskaffe en
	ny tusch.
	\subsection{Udvalg}
	Holdet har også en række udvalg, se \S8.
	\subsection{Indehavere af ansvarsposter}
	\begin{itemize}
		\item Kasserer: Mikkel Hammelev
		\item Tuschansvarlig: Asger Hautop Drewsen
	\end{itemize}


	\section{Holdets time}
	Ved holdets time samles holdet og diskuterer vigtige emner fra den forgangne eller
	fremtidige tid. Holdet bør afholde holdets time mindst én gang om ugen i
	undervisningsperioder.
	\subsection{}
	Alle holdets medlemmer bør i udgangspunktet deltage ved holdets time.
	\subsubsection{}
	Såfremt et holdmedlem ikke har mulighed for at deltage ved holdets time, kan dette vælge at sende en repræsentant i stedet. Repræsentanten kan ikke være et andet holdmedlem. Repræsentanten har den repræsenteredes tale- og stemmeret ved holdets time, og skal under hele holdets time kaldes ved den repræsenteredes navn.
	\subsection{}
	En person som ikke officielt er på holdet eller årgangen kan godt være til stede ved holdets time, men kan kun deltage aktivt hvis vedkommende er repræsentant for en person fra holdet som ikke er til stede. Ida-Marie Schytt Lassen kan altid være repræsentant for Mads Svejstrup, også hvis han er til stede, da det er en af fordelene ved at være i et forhold.
	\subsection{}
	Holdets time bør afholdes på et sted og tidspunkt som er belejligt for størstedelen af holdets medlemmer.
	\subsection{}
	Ved holdets time bør dagsordenen som minimum indeholde:\\
	\begin{itemize}
		\item \KTD
		\begin{itemize}
			\item Valg af ordstyrer
			\item Referent vælges af referentudvalget
		\end{itemize}
		\item Kage
		\item Kage NU (næste uge)
		\item SNAPS 0
		\item IntDeS
		\item Afstemninger
		\item SNAPS (se \S{7})
		\item DD (Dummedragt)
		\item Ævl. (evt.)
	\end{itemize}
	\subsection{}
	Ønsker man at stemme om noget ved holdets time, bringer man en bemærkning herom, og afstemningen vil blive taget op under punktet ”afstemninger”.
	\subsection{}
	Ifm. referatskrivning skal referenten altid benytte samme nummerordensnotation som den der anvendes af ordstyreren på tavlen. Referater skal altid udsendes til resten af holdet i txt-format.
	\subsection{}
	Alle afstemninger til holdets time undtagen ved klandringer er d-ære afstemninger hvor $d \geq 2$.
	\subsection{}
	Til IntDeS fremlægger et medlem af holdet en præsentation om et valgfrit emne. Den person har fremlagt vælger personen til næste uge.
	\subsection{}
	Har man lavet et punkt uden indhold, så skal den person som kom med punktet tage en straf besluttet med farvesystemet fra \JJB.

	\section{SNAPS/klandringer}
	SNAPS er forkortelse for ”Snaps Neutraliserer Alle Pinlige Stunder”. Et holdmedlem kan klandre et eller flere af de andre holdmedlemmer, hvis det føler sig forurettet af disse. Klandringer afgives som udgangspunkt ved starten af punktet ”SNAPS” under holdets time.
	\subsection{}
	Proceduren for klandringer er som følger:
	\begin{itemize}
		\item Alle klandringer fremsiges og opskrives på en tavle
		\item Når alle har fremsagt deres klandringer, behandles hver klandring individuelt.
		\begin{itemize}
			\item Klandreren får først lov til at uddybe sin klandring.
			\item Den klandrede får lov til at fremsige sit forsvar i op til 60 sekunder, medmindre man har fået dispensation hos ordføreren.
			\item Der stemmes om hvorvidt klandringen er rimeligt afgivet.
		\end{itemize}
		\item Resultatet føres til referat.
		\item Evt. yderligere klandringer kan tilføjes mellem klandringer, med ændring i konsekvens jvf. Stk. 8.
	\end{itemize}
	\subsection{}
	Det er ikke tilladt at klandre den samme person flere gange for den samme sag.
	\subsection{}
	Det er dog tilladt at klandre en person for den samme hændelse, hvis hændelsen har fundet sted flere gange.
	\subsection{}
	Det er ikke tilladt at klandre nogen på vegne af flere medlemmer, da dette svare til at klandre flere gange for det samme, jf. stk. 1.
	\subsection{}
	Det er ikke tilladt at klandre nogen som ikke er til stede ved holdets time.
	\subsubsection{}
	Tutorer for holdet - nuværende såvel som tidligere - er undtaget fra denne regel.
	\subsubsection{}
	Undtagelser er tilfælde hvor et medlem to eller flere gange i træk ikke er mødt op ved holdets time.
	\subsection{}
	Taber man en klandring, uanset om man gav eller modtog den, skal man give hhv. den klandrede eller klandreren en øl, sodavand eller tilsvarende, samt betale DKK 5,- til bødekassen.
	\subsection{}
	Det er muligt for et holdmedlem at fremsige en klandring for en hændelse som medlemmet har belæg for at vil komme til at ske i fremtiden. Der stemmes i sådanne tilfælde som normalt. Såfremt hændelsen ikke indtræffer som forventet, og klandringen gik igennem, inverteres resultatet af afstemningen, og straffen fordobles til 2 øl og DKK 10,- til bødekassen.
	\subsection{}
	Fremsiger man en klandring efter uddybningen af den 2. klandring er påbegyndt, modtager man dobbelt straf såfremt kalndringen vurderes ikke at være rimeligt afgivet.
	\subsection{}
	Resultatet for afstemninger noteres altid ved udvidet binær notation, hvor 0 betyder at klandringen var rimeligt afgivet, og 1 betyder at klandringen ikke var rimeligt afgivet. 8 betyder at klandringen var afgivet efter 2. klandrings uddybning var startet, og at den ikke var rimeligt afgivet.
	\subsection{}
	Hver uge uddeles Dummedragten til en person fra holdet som har tabt en klandring. Denne person findes ved en d'ary afstemning, og skal efterfølgende have dragten på til en forelæsning inden næste holdets time.
	\subsection{}
	Undlader en person som har fået tildelt dummedragten at tage den på som vedtægterne foreskriver, skal vedkommende give øl til alle tilstedeværende til næste holdets time.

	\section{Udvalg}
	På dat5 er der en række udvalg som har forskellige ansvarsposter. Se underparagraffer for en liste over udvalg.
	\subsection{Nuværende indehavere af udvalgsposter:}
	\subsubsection{Dummedraftudvalget}
	\begin{itemize}
		\item David Carlos Zachariae
		\item Ulrik Boll Djurtoft
	\end{itemize}
	\subsubsection{Rus-2-tur-udvalget}
	\begin{itemize}
		\item Oskar Haarklou Veileborg
		\item Jacob Hougaard Bennedsen
	\end{itemize}
	\subsubsection{Tour de Fredagsbar-udvalget}
	\begin{itemize}
		\item David Carlos Zachariae
		\item Ulrik Boll Djurtoft
	\end{itemize}
	\subsubsection{Wheel of Death-udvalget}
	\begin{itemize}
		\item Oskar Harklou Veileborg
		\item Steffen Strunge Mathiesen
	\end{itemize}
	\subsubsection{Referentudvalget}
	\begin{itemize}
		\item Tilpas meget Mads Svejstrup
	\end{itemize}
	\subsubsection{Vedtægtsudvalget}
	\begin{itemize}
		\item Asger Hautop Drewsen
		\item Klaus Olesen
		\item Ulrik Boll Djurtoft
	\end{itemize}

	\subsection{Dummedragtudvalg}
	Dummedragtudvalget står for dummedragt.
	\subsection{Rus-2-tur-udvalget}
	Rus-2-tur-udvalget står for at planlægge rus-2-tur.
	\subsection{Tour de Fredagsbar-udvalget}
	Tour de Fredagsbar-udvalget står for at planlægge Tour de Fredagsbar.
	\subsubsection{}
	På hver fredagsbar skal der scores en rus, som skal gang banges, lige meget om høn er en dreng eller pige.
	\subsection{Wheel of Death-udvalget}
	Der skabes et Wheel of Death.
	\subsubsection{Referentudvalget}
	Referentudvalget er ansvarligt for at der bliver ført referat, ved at vælge en referent.
	\subsubsection{Vedtægtsudvalget}
	Vedtægtsudvalgets opgave er at styre det git repository hvor kildekoden til vedtægterne er.


	\section{\KTD}
	Ved f.eks. stemmelighed eller valg af ordstyrer kan man opnå en afgørelse ved hjælp af \KTD. \KTD er et genialt og originalt spil, som intet har at gøre med spillet Sten-Saks-Papir. Reglerne er som følger:
	\subsection{}
	De to deltagere stiller sig over for hinanden og begynder at slå, ligesom i Sten-Saks-Papir. Mens man slår skal man ændre håndtegn når man siger objekternes navne til og med det objekt man har valgt.
	\subsection{}
	Ender en runde uafgjort tager man en runde til, blot med omvendt win-condition: i ulige runder gælder det således om at vinde og i lige runder gælder det om at tabe.

	\section{\JJB}
	\JJB er et elektronisk dødshjul designet og konstrueret af Oskar Harklou Veileborg. Det kan bruges til at indføre midlertidige regler til holdets time.
	\subsection{Regler}
	En regel indført vha. \JJB er gældende resten af den holdets time den er indført + den næste, dog ikke i perioden imellem disse.
	\subsection{Farver}
	\subsubsection{Grøn}
	Grøn betyder at forslaget er gået igennem samt at forslagsstilleren skal have kage med til næste holdets time, ud over den person som blev besluttet ved Kage NU.
	\subsubsection{Gul}
	Gul betyder at forslaget går til normal afstemning og at forslagsstilleren skal bunde en halv øl inden holdets times afslutning, der hvor holdets time afholdes.
	\subsubsection{Rød}
	Rød betyder at forslaget ikke er gået igennem og at forslagsstilleren skal bunde en hel øl inden holdets times afslutning, der hvor holdets time afholdes.


	\vfill
	\doclicenseThis
\end{document}
